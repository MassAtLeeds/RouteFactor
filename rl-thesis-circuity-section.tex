

\subsection{Circuity} \label{scircuity} \index{circuity} \index{Euclidean
distance} \index{route distance}
In practice, the
network of roads, paths and other guideways of the transport system
rarely lead from a to b (or rather
i to j, in our notation) directly. Instead they
form a more or less circuitous path (\cref{fig:routes}). Previous work on this has
been conducted with respect to transport to work. There is strong empirical
evidence that circuity ($Q$) is \emph{not} constant, but varies depending on
the length of trip \citep{Levinson2009} and the structure of the transport
network \citep{parthasarathi2012network}, which varies between countries
\citep{Ballou2002} and continuously over space \citep{Barthelemy2011}.

\begin{figure}[h]
 \centering
 \includegraphics[width=7 cm]{EuclideanDistance}
  \includegraphics[width=7 cm]{NetworkDistance}
 \caption[Schematic of Euclidean and network distances]{Schematic of
 Euclidean and network distances. Thanks to David Levinson, who licensed this
 work, originally published in \citet{Levinson2009} with a Creative Commons licence.}
 \label{fig:routes}
\end{figure}

Regarding
typical values, $Q$ values between 1.21 and 1.23 have been reported for
walking trips to rail stations in Calgary, Canada \citep{O'Sullivan1996}.
\citet{Levinson2009} analysed the circuity of 5,000 home-work trips in and
around Portland, USA, and found an average circuity of 1.18 overall. In the
same study, it was also confirmed that
circuity is highly dependent on the distance travelled: for 50,000 random
point-pairs, circuity decreased from 1.58 to 1.2 as the distance increased from
5 km and less to over 45 km. Based on these results, a preliminary analysis
suggests that the relationship is logarithmic (\cref{fig:circuity}).
Circuity (referred to as a ``detour index'') was reported by \citet[p.~565]{cole1968quantitative}
for 12 districts in England, Scotland and Wales. Values ranged from
1.17 (in Somerset) to 2.19 (Aberdovey); the mean was 1.4 overall.

This result was corroborated by \citet{Ballou2002}, who
found an average circuity of 1.4 for England
as a whole, based on a sample of 37 points. Other than \citet{Levinson2009},
none of these studies
included the impact of distance on average circuity
values, instead reporting single values for entire areas.
\citet{Levinson2009} provide strong evidence to suggest that
circuity, taken as an average value over hundreds of measurements,
actually declines with distance, in a way that would be compatible
with all the previously mentioned estimates of circuity.

\begin{figure}[h]
 \centering
 \includegraphics[width=14 cm]{circuity.pdf}
 % circuity.png: 550x450 pixel, 72dpi, 19.40x15.88 cm, bb=
 \caption[The decay of circuity with distance travelled]{The decay of circuity
with distance travelled. Data from
\citep{Levinson2009}, plotted here with a logarithmic decay ($y = a +
b*log(x)$), where $a$ = 1.72 and $b$ = -0.14. Coefficients calculated using the
command nls in R.}
 \label{fig:circuity}
\end{figure}

Analysis of the results from \citet{Levinson2009} suggest that $Q$ decays
logarithmically with increasing distance (see \cref{fig:circuity}):
\begin{equation}
Q = a + b \times log(dE)
\label{eq:circ}
\end{equation}
where a and b are coefficients calculated to be 1.72 and -0.14, respectively,
based on the \citet{Levinson2009} paper. Of course, using the results of
a US study as the basis for assumptions in the UK is no guarantee that the
assumptions will hold in practice, especially when
$Q$ varies from country to country (and almost certainly at lower levels also,
depending on the local road network and proximity of impassable obstacles
such as rivers, railways and motorways). There is additional support
for $Q$ decaying with increasing $dE$ from 
theoretical sources \citep{Barthelemy2011}.
The evidence reviewed suggests that, if one must
assume that $dR = f(dE)$ (as is the case here, as only Euclidean distances are
provided in the census data), \cref{eq:circ} is likely to
provide a more accurate description of reality than assuming that $dR = dE$.
The principle of Occam's razor states that the simplest solution that
fits the data should generally be preferred. In this case
recent evidence shows that $dR = dE$ simply
does not fit the data, so $Q = 1.7 + -0.14 \times log(dE)$
is used here. If a single circuity factor is required, Ballou's (2002) estimate
of 1.4 for the UK is recommended, especially as this coincides
with the circuity value interpolated in \cref{fig:circuity} around the 10 km
mark, roughly the median distance travelled to work in the UK.

Of course, circuity is affected by many other variables in addition to
Euclidean distance. In addition, it is wrong to assume that
more circuitous paths are always more energy intensive, as a complex
range of factors combine to determine the most energy efficient path
to take at any particular time \citep{Ericsson2006}. There are also large
inter-modal variations in circuity: pedestrians
and cyclists have been found to have particularly low $Q$ values \citep{Iacono2010}.
It can be expected that public transport users must endure longer route lengths
due to the need to get to and from train stations, bus stops and other
nodes to join the network, whereas cars and cycles can join almost anywhere.
In addition, it would be possible to weight $Q$ area by area, based on local
estimates of \emph{global accessibility} (see \cref{skeyconcepts})
that can could be computed by calculating the difference between $dR$ and
$dE$ for randomly (or intelligently) selected origin-destination pairs.

Beneficial as this process would be, yet these factors still omit
the impact of car park proximity, car sharing,
and multi-mode trips: in a more complex (potentially agent-based) model these
could conceivably be included.
For the time being it is assumed that \cref{eq:circ} holds for
all trips of the same distance: quantitative evidence of the impact of other
factors is scarce. If more data to weight $Q$ by other factors such
as mode emerges, the model should be updated.

